\documentclass[a4paper,12pt]{article}

\usepackage{paralist}
\usepackage[utf8x]{inputenc}
%\usepackage[T1]{fontenc}
\usepackage{eurosym}
\usepackage{fancyhdr}
\usepackage[italian]{babel}
\usepackage{url}
%\usepackage{lineno}
\usepackage{wrapfig}
\usepackage[pdftex]{graphicx}
%\usepackage{graphicx}
%\usepackage{sidecap}

\topmargin=-1cm
\oddsidemargin=0cm
\evensidemargin=0cm
\textwidth=15cm
\textheight=22cm

%\headsep=-0.5cm
% \headheight=0cm

\pagestyle{fancyplain}
%\markboth{}


%\headheight=1cm
% %\footskip=6cm
% \rhead[]{}
% \chead[]{\includegraphics[width=.8\textwidth]{vers_piccola_una_riga_nero.jpg}}
% %, \rhead, \lfoot, \cfoot and \rfoot

\lfoot{\medskip Andrea Trentini\\
andrea.trentini@gmail.com\\
Tel. +39 3356671695}
\cfoot{\hrule}
\rfoot{\medskip Via Molinazzo, 15\\
20032, Cormano (MI)\\
Italia}

\title{\textbf{Proposta\\
per la creazione di un\\
``Hacklab'' a Cormano}}
%\author{Andrea Trentini}
\date{Febbraio 2016}

%%%%%%%%%%%%%%%%%%%%
%%%%%%%%%%%%%%%%%%%%
\begin{document}
\maketitle
\thispagestyle{fancy}
%\floatstyle{boxed}
%\restylefloat{figure}

\hspace{.40\textwidth}\textbf{Alla cortese attenzione del Sindaco}

\hspace{.40\textwidth}\textbf{Tatiana Cocca}
% tatiana.cocca@comune.cormano.mi.it

%\hspace{.40\textwidth}\textbf{Marco Pilotti (assessore)}
%% marco.pilotti@comune.cormano.mi.it

%\hspace{.40\textwidth}\textbf{Fabrizio Vangelista (assessore)}
%% fabrizio.vangelista@comune.cormano.mi.it

%\hspace{.40\textwidth}\textbf{Annamaria Arcidiacono (ufficio stampa)}
%%  annamaria.arcidiacono@comune.cormano.mi.it

\medskip

\hspace{.50\textwidth}\textbf{Comune di CORMANO}

\hspace{.50\textwidth}Piazza Scurati, 1

\hspace{.50\textwidth}20032 Cormano (MI)



% \tableofcontents
% \medskip
% \hrule


\section*{Introduzione}

In Luglio 2015 è stata organizzata una serata ``Arduino'' presso la Biblioteca 
Comunale di Cormano. L'autore della presente era lo \textit{speaker}.
L'incontro è durato poco più di un paio d'ore e nonostante il caldo torrido 
(era il periodo di ``Acheronte'') la partecipazione si è attestata sulla 
cinquantina di persone coinvolte, attente e partecipative... tanto che la 
sessione domande e risposte ha preso circa metà della serata.

A seguito dell'interesse espresso dai partecipanti alla serata è stato successivamente organizzato un mini corso di 8 incontri serali (a fine 2015), sempre ospitato presso la Biblioteca Comunale. L'affluenza è stata così elevata da dover fissare un numero chiuso (30 iscritti).
I feedback del corso sono stati quasi tutti positivi, le note negative riguardavano la struttura (mancanza di un proiettore) e la sottovalutazione dei prerequisiti per la partecipazione.
Tutti i partecipanti al corso hanno espresso il desiderio di continuare con ``qualche tipo di attività'' a tema.
\textbf{Anche contribuendo finanziariamente.}

\section*{Proposta ``Hacklab Cormano''}

Sull'onda di tale entusiasmo lo scrivente propone al Comune di Cormano l'avallo formale e il supporto operativo (fornendo un locale per i ritrovi periodici) alla creazione di una \textbf{associazione} (non a fini di lucro) dedita a studio e sperimentazione di tecnologie aperte e piattaforme Arduino e simili.

L'associazione, al posto dell'organizzazione di ``semplici'' corsi a pagamento, avrebbe il vantaggio di radunare persone appassionate ed interessate a partecipare e a \textbf{contribuire fattivamente} alla condivisione di conoscenza sul tema. Inoltre stabilirebbe un rapporto più equilibrato fra i membri: \textit{blended learning} invece del trito rapporto docente-discente. E darebbe potere decisionale ai membri attraverso l'assemblea.
Ovviamente lo scrivente fornirebbe la prima e massiccia iniezione di conoscenze attraverso una serie di lezioni tradizionali (gli argomenti sarebbero i medesimi del corso del 2015) nell'ottica di portare i membri all'indipendenza il più presto possibile.

Una nota terminologica: i termini ``hacking'', ``hacker'' e ``hacklab'' sono spesso interpretati scorrettamente in Italia, non vanno associati ai crimini informatici ma alla semplice passione per lo studio degli \textit{internals} delle cose.
Però qualora si preferisse un termine meno interpretabile (o meno schierato politicamente) si possono usare ad esempio termini più soft  e/o italianizzati come:\\
- ``Arduino lab'' (col difetto di essere associato ad una piattaforma particolare)\\
- ``Makerspace''\\
- ``Laboratorio makers''\\
- ...


\textbf{Lo scrivente farebbe da promotore e fondatore dell'iniziativa, stilerebbe uno statuto, un manifesto e un atto costitutivo da proporre agli eventuali firmatari.}

\section*{Proposta operativa}

Operativamente si potrebbe organizzare (anche nel breve termine) una serata di presentazione dell'idea e raccogliere feedback in merito, sulla base di questo incontro decidere se portare avanti l'ipotesi di fondazione dell'associazione o meno.

Il Comune di Cormano dovrebbe in questo caso ospitare la serata e fare un mailing per la divulgazione dell'evento.

Inoltre il Comune dovrebbe decidere se intende dare disponibilità di un locale (biblioteca?) e con che termini (giorni della settimana, orari, etc.) in modo da avere il quadro completo per la decisione.
Nel caso il numero di adesioni fosse sufficiente da avviare l'associazione lo scrivente ritiene che un regime ragionevole di ritrovi sia \textbf{quindicinale}, l'eventuale locale verrebbe quindi occupato una sera ogni due settimane. Eventualmente per i primi tempi con una frequenza maggiore, ma mai più di una volta la settimana.



% \medskip
% \hrule
% \medskip

\medskip

\hfill Grazie, a presto!

\hfill Andrea Trentini

%\begin{figure}
% \centering
%\hfill \includegraphics[width=5cm]{../../Firma.png}
 % Firma.png: 640x320 pixel, 72dpi, 22.58x11.29 cm, bb=0 0 640 320
%\end{figure}



%%%%%%%%%%%%%%%%%%%%
\end{document}
