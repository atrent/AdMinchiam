\documentclass[a4paper,11pt,oneside]{article}

\usepackage[italian]{babel}
\usepackage[utf8]{inputenc}
\usepackage{lineno}
\frenchspacing

% Impostazioni per uso bollo
\newcommand{\bollo}{
\pagestyle{empty}
\widowpenalty=0
\raggedbottom
\setlength{\baselineskip}{.5cm}
\setlength{\parindent}{0pt}
\setlength{\medskipamount}{0pt}
\setlength{\bigskipamount}{0pt}
\setlength{\smallskipamount}{0pt}
}

% Impostazioni per carta uso bollo
\usepackage[body={132mm,240mm},top=33.6mm,left=28.8mm,headsep=0mm,nohead]{geometry}

% Impostazioni numerazione in enumerate
\renewcommand{\labelenumi}{\alph{enumi})}
\renewcommand{\labelenumii}{\arabic{enumii}.}

% Impostazione interlinea doppia per carta uso bollo (aggiustare)
%\linespread{2}
\linespread{1.5}

\begin{document}

\pagestyle{empty}
\linenumbers
\textbf{Statuto sociale dell'Associazione ``FIXME''}

\section{Disposizioni generali}

\subsection{Denominazione}
É costituita l'Associazione Culturale ``FIXME'', d'ora in poi denominata semplicemente Associazione.

L'Associazione non ha finalità di lucro e gli eventuali utili conseguiti dovranno essere utilizzati per il conseguimento degli scopi istituzionali.

\subsection{Oggetto}
Scopo dell'Associazione è quello di FIXME

% e.g: promuovere la diffusione della cultura musicale, con particolare attenzione alla musica tradizionale per arpa celtica.

L'Associazione intende perseguire i propri scopi tramite le seguenti attività:

\begin{enumerate}
	\item FIXME
% e.g.
%   \item   corsi di studio e formazione musicale; 
%   \item   registrazione e pubblicazione di materiale audio;
%   \item   utilizzo dei mezzi di comunicazione radio e televisivi e le reti telematiche in maniera funzionale ai propri scopi istituzionali;
%   \item   svolgimento di attività di collaborazione nei confronti di istituzioni, associazioni o persone fisiche impegnate nella diffusione e nello sviluppo della cultura musicale. 
\end{enumerate}

\subsection{Durata}
L'Associazione ha durata illimitata e può essere sciolta con una delibera dell'Assemblea dei soci in riunione straordinaria.

\section{I Soci}

\subsection{Composizione dell'Associazione}
Il numero dei soci è illimitato e può diventare socio chiunque si riconosca nel presente statuto, indipendentemente dalla propria appartenenza politica e religiosa,
sesso, cittadinanza, appartenenza etnica e professione. Agli aspiranti soci sono richiesti l'accettazione dello statuto, il godimento di tutti i diritti civili e il rispetto della civile convivenza.

\subsection{Domanda di ammissione}
Chi intende aderire alla Associazione deve rivolgere espressa domanda all'Assemblea dei soci recante la dichiarazione di condividere le finalità associative e l'impegno ad accettarne e osservarne Statuto e Regolamenti.

\subsection{Soci}
Sono soci tutti coloro che si riconoscono nei fini della Associazione, che sono disposti a sostenerla economicamente per il raggiungimento degli scopi istituzionali, che presentano domanda di ammissione alla Associazione e che vengono accettati.

I soci si impegnano al pagamento della quota sociale prevista e stabilita annualmente dall'Assemblea dei soci.

\subsection{Accettazione del socio}
L'ammissione del socio deve essere approvata all'unanimità dall'Assemblea dei soci, convocata in seduta straordinaria per deliberare in merito. In  caso di diniego, l'Assemblea non è tenuta a specificarne la motivazione.

\subsection{Diritti e doveri del socio}
Tutti i soci hanno gli stessi doveri e godono degli stessi diritti nei confronti dell'Associazione. I soci hanno diritto a:

\begin{enumerate}
\item  frequentare i locali a disposizione dell'Associazione;
\item  partecipare a tutte le iniziative e manifestazioni promosse dall'Associazione;
\item  riunirsi in assemblea per discutere e votare sulle questioni riguardanti l'Associazione; 
\item  eleggere ed essere eletti membri degli organi dell'associazione. 
\end{enumerate}

Hanno diritto di voto in Assemblea tutti i soci regolarmente iscritti.

\subsection{Recesso del socio}
La qualifica di socio si perde per:

\begin{enumerate}
 \item  mancato pagamento della quota sociale; 
 \item  espulsione o radiazione; 
 \item  dimissioni, che devono essere presentate per iscritto all'Assemblea dei soci. 
 \item  decesso; 
\end{enumerate}

\subsection{Esclusione del socio}
L'Assemblea dei soci ha la facoltà di intraprendere azione disciplinare nei confronti del socio, mediante - a seconda dei casi - il richiamo scritto, la sospensione temporanea o l'espulsione o radiazione per i seguenti motivi:

\begin{enumerate}
 \item  mancato rispetto delle disposizioni dello statuto, di eventuali regolamenti o delle deliberazioni degli organi sociali; 
 \item  denigrazione dell'Associazione, dei suoi organi sociali, dei suoi soci; 
 \item  intralcio al buon andamento dell'Associazione, ostacolandone lo sviluppo e perseguendone lo scioglimento; 
 \item  commissione o provocazione gravi disordini durante le assemblee; 
 \item  appropriazione indebita dei fondi sociali, atti, documenti od altro di proprietà dell'Associazione; 
 \item  arrecazione di danni morali o materiali all'Associazione, ai locali ed alle attrezzature di sua pertinenza. In caso di dolo, il danno dovrà essere risarcito. 
\end{enumerate}

\subsection{Ricorso}
Contro ogni provvedimento di sospensione, espulsione o radiazione, è ammesso il ricorso entro trenta giorni, sul quale decide in via definitiva la prima assemblea utile dei soci.

\section{Patrimonio sociale, entrate e rendiconto}

\subsection{Patrimonio sociale}
Il patrimonio sociale dell'Associazione è indivisibile ed è costituito da:

\begin{itemize}
 \item  contributi, erogazioni e lasciti diversi; 
 \item  fondo di riserva. 
\end{itemize}

L'utilizzo del fondo di riserva è vincolato alla decisione dell'Assemblea dei Soci.

\subsection{Entrate}

Per l'adempimento dei suoi compiti la Associazione dispone delle seguenti entrate:

\begin{enumerate}
\item versamenti effettuati dai fondatori originari, dei versamenti ulteriori effettuati da detti fondatori e da quelli effettuati da tutti coloro che aderiscono all'Associazione;
\item introiti realizzati nello svolgimento della propria attività.
\end{enumerate}

L'Assemblea dei soci stabilisce annualmente la quota di iscrizione all'Associazione. 

I versamenti al fondo di dotazione possono essere di qualsiasi entità, fatti salvi i versamenti minimi come sopra determinati, per l'iscrizione annuale. 

Qualsiasi tipo di versamento non crea altri diritti di partecipazione o quote indivise di partecipazione trasmissibili a terzi.

Tutti i versamenti effettuati all'Associazione, comprese le quote di iscrizione, sono a fondo perduto e non sono rivalutabili in nessun caso, nemmeno in caso di scioglimento, morte, estinzione, recesso o esclusione dalla Associazione.

\subsection{Rendiconto}
Il rendiconto comprende l'esercizio sociale dal 1 gennaio al 31 dicembre di ogni anno e deve essere presentato all'assemblea dei soci entro il 30 aprile
successivo. Ulteriore deroga può essere prevista in caso di comprovata necessità o impedimento.

Il rendiconto dovrà essere composto da un documento che illustri e riassuma la situazione finanziaria dell'Associazione, con particolare riferimento allo stato del fondo di riserva.

\section{L'Assemblea}

\subsection{Composizione}
L'Assemblea, Ordinaria e Straordinaria, è l'organo deliberativo dell'Associazione; hanno diritto a parteciparvi tutti i soci regolarmente iscritti.

\subsection{Competenze dell'Assemblea}
L'Assemblea Ordinaria ha le seguenti competenze:

\begin{enumerate}
 \item  approvare il rendiconto economico e finanziario; 
 \item  approvare le linee generali del programma di attività ed il relativo documento economico di previsione; 
 \item  eleggere gli organi dell'Associazione alla fine del mandato o in seguito alle dimissioni degli stessi, votando la preferenza a nominativi scelti tra i soci;
 \item  deliberare su tutte le questioni attinenti la gestione sociale.
\end{enumerate}

\subsection{Convocazione dell'Assemblea}
L'Assemblea ordinaria viene convocata una volta all'anno nel periodo che va dal 1 gennaio al 30 aprile. 

L'Assemblea straordinaria viene convocata tutte le volte che l'Assemblea dei soci lo reputi necessario o ogni qual volta ne faccia richiesta almeno un quinto dei soci.

L'Assemblea dovrà aver luogo entro trenta giorni dalla data in cui viene richiesta; la convocazione avviene mediante semplice comunicazione verbale o per posta elettronica indirizzata ai singoli Soci.

L'avviso di convocazione è spedito almeno dieci giorni prima dell'Assemblea, e indica il luogo, la data e l'ora in cui si terrà l'Assemblea stessa, con il relativo ordine del giorno.

\subsection{Costituzione dell'Assemblea}
L'Assemblea, sia ordinaria che straordinaria, è regolarmente costituita alla presenza della metà più uno dei soci.

\subsection{Delibere Assembleari}
L'Assemblea, sia ordinaria che straordinaria, delibera a maggioranza semplice (la metà più uno) sull'insieme dei soci presenti. Le votazioni in Assemblea ordinaria e straordinaria avvengono per alzata di mano o per appello nominale.

\subsection{Eccezioni alle delibere}
Per delibere riguardanti modifiche allo Statuto o ai Regolamenti è indispensabile la presenza di almeno il 50\% dei soci ed il voto favorevole di almeno tre quinti dei partecipanti.

\subsection{Verbalizzazione}
L'Assemblea, all'inizio di ogni sessione, elegge tra i soci presenti un presidente e un segretario di assemblea: il segretario provvede a redigere i verbali delle deliberazioni, verbali devono essere sottoscritti dal presidente dell'Assemblea e dal segretario.

I verbali dell'Assemblea sono messi a disposizione dei soci presso la sede sociale dell'Associazione.

\section{Gli organi sociali}

\subsection{Organi dell'Associazione}
Gli organi dell'Associazione sono:

\begin{itemize}
 \item  il Presidente: ha la rappresentanza legale dell'Associazione ed è il responsabile di ogni attività della stessa;
 \item  il Segretario: cura gli aspetti amministrativi dell'Associazione. 
\end{itemize}

L'Associazione può inoltre distribuire fra i suoi componenti altre funzioni attinenti a specifiche esigenze legate alle proprie attività.

\subsection{Compiti del Presidente}
Compiti del Presidente sono:

\begin{itemize}
 \item  eseguire le delibere dell'assemblea;
 \item  formulare i programmi di attività sociale sulla base delle linee approvate dall'Assemblea e del relativo documento economico di previsione; 
 \item  stipulare tutti gli atti e i contratti inerenti le attività sociali;
 \item  decidere le modalità di partecipazione dell'Associazione alle attività organizzate da altre Associazioni ed Enti, e viceversa, se compatibili con i principi ispiratori del presente Statuto. 
\end{itemize}

\subsection{Compiti del Segretario}
Compiti del Segretario sono:

\begin{itemize}
 \item  predisporre il rendiconto economico e finanziario consuntivo;
 \item  predisporre il documento economico di previsione;
\end{itemize}

\subsection{Collaborazioni esterne}
L'Assemblea dei soci può avvalersi, per compiti operativi o di consulenza, di commissioni di lavoro da esso nominate, nonché dell'attività volontaria di cittadini non soci, in grado, per competenze specifiche, di contribuire alla realizzazione di specifici programmi.

\section{SCIOGLIMENTO DELL'ASSOCIAZIONE}

\subsection{Scioglimento dell'Associazione}
La decisione motivata di scioglimento dell'Associazione deve essere presa da almeno i quattro quinti dei soci aventi diritto al voto, in un'Assemblea valida alla presenza della maggioranza assoluta dei medesimi.

L'Assemblea decide sulla devoluzione del patrimonio residuo, dedotte le eventuali passività, per uno o più scopi stabiliti dal presente Statuto e comunque per associazioni con finalità analoghe o ai fini di pubblica utilità, procedendo alla nomina di uno o più liquidatori scelti preferibilmente tra i soci.

\section{DISPOSIZIONI FINALI}

\subsection{Rimando}
Per quanto non previsto dallo Statuto o dal Regolamento interno, decide l'Assemblea ai sensi del Codice Civile e delle leggi vigenti.

\end{document}
